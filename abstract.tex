\expandafter\ifx\csname ifdraft\endcsname\relax
\documentclass[11pt]{jsreport}
\usepackage{mypackage}
\begin{document}
\fi


\begin{abstract}
  %英語の部分
\begin{comment}
    
  To improve the reliability of nano-satellites, it is essential 
  to fix satellite failures which are generated in design phase and
  manufacturing phase.
  Nevertheless, fault analysis depends on human ability or experiences, 
  and the process of figuring out possible failures and
  identifying fault location requires considerable 
  human resources and time.
  This research proposes a new approach to support identifying failure causes
  of satellite by showing the verify methodology and the performance index which is  
  searched based on the model of signal, electrical and physical interaction between
  subsystem or components and models of signal transmission routes in a satellite.
  The effectiveness of the approach
  is proved by applying it to a simple satellite model and comparing 
  with results or process of identifying failure cause without the approach. \\
\end{comment}

近年,大学や高専などの教育機関や,民間企業による超小型衛星の
開発,およびそれを利用した事業の展開が盛んになっている
一方で,超小型衛星の信頼性の低さが問題となっている.%修正
軌道上故障に関する調査の結果
信頼性の低さの原因として,設計および製造過程における不良が多いことが分かっており,
地上試験によって不具合の改修,対策を十分に行うことが超小型衛星の信頼性
向上のために重要である.
 一方で,衛星のように多くの機器が複雑に絡みあったシステムでは,
一つの不具合事象に対して非常に多くの故障が考えられ,不具合事象から
故障箇所の特定を行うことは非常に多くの知識と経験を必要とする.
 そのため,衛星開発を専門としない機関などの
 経験が浅いエンジニアや衛星に関する知識の乏しいエンジニアが
 不具合事象から網羅的に故障候補を
 洗い出し,故障箇所の特定を行うことは困難である.%言葉の定義
 %もしくは故障箇所
 これに対して,機器や故障状態をオントロジーなどを用いてモデル化し,
 不具合事象から故障仮説を網羅的に洗い出すための研究が広く行われている.
 一方で,故障仮説の検証段階に対して取り組んだ研究は少なく,検証過程が
 人の知識や経験に依存してしまっている.
 
 そこで,本研究ではコンポーネント間の接続関係モデル,衛星内の情報
 伝達の経路モデルを用いて衛星の故障候補の
 検証手順(打つべきコマンド,確認事項)を探索し,
 それらをコマンドの安全性及び,故障候補切り分け能力を示す指標と共に提示する
 ことで不具合分析を支援する手法を提案する.
 
 本手法を用いて,簡易的な衛星モデルに対して不具合分析を実践することで,
 コマンドによる故障箇所の特定作業が体系化できること,設計不備の発見につながることを確認した.
% また,本手法を用いた不具合分析で扱うことのできる故障に限界があることがわかり,今後の方針を示す.

\end{abstract}


\expandafter\ifx\csname ifdraft\endcsname\relax
  \end{document}
\fi