\expandafter\ifx\csname ifdraft\endcsname\relax
\documentclass[11pt]{jsreport}
\usepackage{mypackage}
\begin{document}
\fi

\chapter{本手法による不具合分析の実践と評価}

\section{概要}
不具合分析の具体的な流れをみるために,いくつかの事例を取り上げて実践した結果を示す.
以下では,まず実際に故障箇所特定が行えた事例,特定できなかった事例を取り上げ,そこから得られた知見に関して述べる.
次に,評価指標に関する考察を行うために,優先する指標を変えることによって故障特定のプロセスに変化が現れるのか
どうかを検証した結果に関して述べる.
%ここら辺もう少し見るところがある気がする.
最後に,本手法によって扱うことのできる故障の種類に関する限界を述べ,発展させるために必要な
方針に関して議論を行う.

\section{実践例}

%もう少しsectionわけの名前を考える.
\subsection{ヒータ接触不良の特定}
まず,以下の図\ref{fig:fault_mode1}のような故障を考え,
不具合分析を行っていく.\\
%図の配置を考える.
\begin{figure}[H]
   \centering
      \includegraphics[height=13.0cm]{figure/fault_mode1.png}
      \caption{故障箇所:リンク32(推進系ヒータ−推進系間)の時の故障候補}
      \label{fig:fault_mode1}
\end{figure}

%問題設定の与え方として,各コンポーネントの状態はどのようになっているかを示したほうが良くないか?
図\ref{fig:fault_mode1}に示すような
推進系ヒータ‐推進系間でのヒータ接触不良が発生している場合を考える.
この時,異常検知の際の不具合事象としては,
\begin{itemize}
   \item 推進系ヒータONコマンド(ID:14)を送信したのに,推進系温度(ID:17)が上昇しない
\end{itemize}
という事象である.
以下に,この事象に本手法を適用した例を示す.
まず,図\ref{fig:tel_phase}に
故障候補の決定及び,テレメトリ情報を用いた確認の段階を示す.
故障候補の決定では,不具合事象を検知するきっかけとなったコマンドとテレメトリが形成する
経路を探索し,targetTEL,targetCOMとして提示している.\\
その後,時間変化するテレメトリ情報を用いて確認できる故障候補を提示し,
返ってくるテレメトリが正常(OK)か否(NG)かを入力させることで,切り分けを行っている.
今回の例ではMOBC及びTOBCは正常に動作しているはずなので,MOBCカウンタ及びTOBCカウンタは正常(OK)
と入力し,それを元に状態の更新を行っている.
\begin{figure}[H]
   \centering
      \includegraphics[height=5.0cm]{figure/COM14_TEL17_show_TEL_list.png}
      \caption{テレメトリによる確認}
      \label{fig:tel_phase}
\end{figure}
次に,図\ref{fig:ini_COM_phase}に示すのが,不具合発生時に送信していたコマンド情報から
考えられるテレメトリの変化を用いて故障候補の確認を行う段階である.
今回は,初期コマンドとしては異常検知の際に送ったコマンド(推進系ヒータON)のみ
を考えている.
確認可能性の高い経路を形成するテレメトリから順に表示され,人間に確認をさせているのが分かる.
ここでの確認テレメトリに関しても,今回の例では正常であるため,そのように入力し状態の更新
を行っている.
%もう少し説明が必要そう
ここに検証しているリンクがどれに当たるのかに関して可視化できるようにしたい
\begin{figure}[H]
   \centering
      \includegraphics[height=7.0cm]{figure/COM14_TEL17_initial_COM.png}
      \caption{初期コマンドを用いた確認}
      \label{fig:ini_COM_phase}
\end{figure}

最後に,以下の図\ref{fig:COM_candidate}に示すのが,上記の流れを経て残った
故障候補を確認できるコマンドを探索し,指標と共に提示した結果である.
残った故障候補は,
\begin{itemize}
   \item コマンドリンク32:推進系ヒータ‐推進系間
   \item テレメトリリンク40:推進系‐推進系温度計間
   \item テレメトリリンク24:推進系温度計‐TOBC間   
\end{itemize}
である.
この時,探索結果として表示されたのはコマンド13(パネルヒータON)と18(推進系ヒータOFF)
であり,これらのコマンドに関する指標が図\ref{fig:COM_candidate}のように示されている.\\
図中においてCheck link numberが「確認可能リンク数」,Mean Probability of checkが「平均確認可能性」,
total\_COM\_numberが「検証コマンド総数」を表している.
またコマンドの衛星生存性への副作用を示す指標に関しては,
impact TEL numが「コマンドによって影響を受けるテレメトリの数」,Remaining PowerとPower consume by this COM
が「コマンド送信前のバッテリ残量とコマンド送信による消費電力」,Attitudeが「姿勢変化を起こすか否か」を示している.
Attitudeに関しては,姿勢変化を起こす場合は"Change",起こさない場合は"Keep"と表示するようにしている.\\
今回の例では,推進系温度計には故障はないはずなので,パネルヒータによって推進系に伝わった熱を読み取り
テレメトリに温度上昇として現れることになる.最終的に残ったリンクが32となり,想定した故障箇所の特定
が行えた.
\begin{figure}[H]
   \centering
      \includegraphics[width=15.0cm]{figure/COM14_TEL17_COM_phase.png}
      \caption{コマンドの選択肢表示及び検証過程}
      \label{fig:COM_candidate}
\end{figure}
%結果に関して何か..

\subsection{評価指標に関する考察}
%ここの文言も微妙
上で示した例(ヒータ接触不良)に関して,コマンドを選択する際に優先する評価指標によって
検証プロセスの違いを示し,評価指標が持つ意味を考察する.


\begin{comment}
   
以下では,2章で示したコマンドの評価指標に関して,複数のコマンド候補があった際に
優先するコマンドの評価指標によって検証結果が変化するかどうかを検証し,評価指標の
定性的な意味を考察した.\\
以下の図\ref{fig:fault_mode3}のような複数の故障がある場合を考える.
%これに関して述べていたか確認する
なお,本手法では1つの故障しか正確に見ることができない手法になっているが,
今回は評価指標による違いを見るためにこのような問題設定を与えた.
\begin{figure}[H]
   \centering
      \includegraphics[height=13.0cm]{figure/fault_mode3.png}
      \caption{故障箇所:リンク33(パネルヒータ−パネル間),リンク37(パネルヒータ−推進系間)の時の故障候補}
      \label{fig:fault_mode3}
\end{figure}

まず,テレメトリによる確認事項の提示と,初期コマンドによって確認できる
項目の提示に関する検証プロセスを以下の図\ref{fig:fault_mode3_TEL_list},\ref{fig:fault_mode3_iniCOM}に示す.
%これは上に入るはず
\begin{figure}[H]
   \centering
      \includegraphics[height=5.0cm]{figure/COM13_TEL16_17_14start_show_TEL_list.png}
      \caption{取得テレメトリによる確認事項の提示及び検証結果}
      \label{fig:fault_mode3_TEL_list}
\end{figure}
\begin{figure}[H]
   \centering
      \includegraphics[height=8.0cm]{figure/COM13_TEL16_17_14start_initial_COM.png}
      \caption{初期コマンドを用いた確認}
      \label{fig:fault_mode3_iniCOM}
\end{figure}

%ここで言うてること,PROP-HTR_PROP間のときでも同じ気がする.とりあえず一回で大きく確認できるのか,
%それとも一回目に残りの全体を見て,その中に故障箇所があることを確定させるのか..
%同じ例に関して考察を述べればいい気がする.特にそれ以外は必要ない
この結果残る故障候補として以下のようになり
\begin{itemize}
   \item 
\end{itemize}
その結果以下のようなコマンド探索結果が提示される.
評価指標を比較すると,平均確認確率は
検証コマンド総数は
となる.
以下では,この2つの指標に関して,優先する指標によってどのように検証プロセスが変化するかを示す.\\
\end{comment}

提示されたコマンドにおいて図\ref{fig:COM_candidate}のように,
「平均確認可能性」は13が高く,「検証コマンド総数」は18が少なくなっている.
13から検証した場合,初めにテレメトリリンク40,24の正常が確認できており,
1つのコマンドで1つのリンクにまで絞り込めている.最終的にリンク32の異常を確認して
故障箇所を特定している.\\
一方で,図\ref{fig:COM_phase_18start}のように
18から検証した場合は1つ目の検証では状態を確認できたリンクはなく,
2つ目のコマンドで正常リンクと異常リンクの切り分けができている.
\begin{figure}[H]
  \centering
  %写真変える.
    \includegraphics[width=13.0cm]{figure/COM14_TEL17_COM_COM_18start.png}
    \caption{コマンド18からの検証プロセス}
    \label{fig:COM_phase_18start}
\end{figure}
運用時,通信が不安定であり不具合分析に使える時間が不定である時は
一度のコマンドで多くの確認ができることが望ましいため,図\ref{fig:COM_candidate}のような
検証プロセスが良いと言える.



複数故障を考えた場合,指標の選択基準によって
探索プロセスが変化する.


平均確認可能性が高いコマンドから送ると,一回のコマンドで多くの絞り込みが行えるので,
時間制約に不確定性があり,コマンドを1度送ることができるかどうかという状況であれば
ワンチャンスをものにできるので良い.

一方で検証コマンド総数を指標に選んだ場合,少ないコマンド数で絞り込みを行えるが,
あくまで最後まで検証を行うことを元にコマンドの数を計算しているため,
結果によっては最悪値の数のコマンドを送信する必要があり,時間制約の厳しいときには十分に
絞り込みを行えないまま検証作業を終えることになる.

%これもう一回やってみる?コマンド数の計算がバグってるから大変
恐らくOBC系列が死んだ時,かつそれに関するコマンドが複数ある場合に有効になりそう.


\begin{comment}
   
図\ref{fig:fault_mode1}に示すような,カメラヒータ−カメラ間
接触不良が発生している場合を考える.
この時,異常検知の際の不具合事象としては,
\begin{itemize}
   \item カメラヒータONコマンド(ID:15)を送信したのに,カメラ温度(ID:18)が上昇しない
\end{itemize}
という事象である.
以下に,この事象に本手法を適用した例を示す.
まず,図\ref{fig:tel_phase}に
故障候補の決定及び,テレメトリ情報を用いた確認の段階を示す.
故障候補の決定では,不具合事象を検知するきっかけとなったコマンドとテレメトリが形成する
経路を探索し,targetTEL,targetCOMとして提示している.\\
その後,時間変化するテレメトリ情報を用いて確認できる故障候補を提示し,
返ってくるテレメトリが正常かどうかを入力させることで,切り分けを行っている.

\begin{figure}[H]
   \centering
      \includegraphics[height=5.0cm]{figure/Tel_phase.png}
      \caption{テレメトリによる確認}
      \label{fig:tel_phase}
\end{figure}
次に,図\ref{fig:ini_COM_phase}に示すのが,不具合発生時に送信していたコマンド情報から
考えられるテレメトリの変化を用いて故障候補の確認を行う段階である.
今回は,初期コマンドとしては異常検知の際に送ったコマンド(カメラヒータON)のみ
を考えている.
%もう少し説明が必要そう
ここに検証しているリンクがどれに当たるのかに関して可視化できるようにしたい
\begin{figure}[H]
   \centering
      \includegraphics[height=7.0cm]{figure/initial_COM_phase.png}
      \caption{初期コマンドを用いた確認}
      \label{fig:ini_COM_phase}
\end{figure}
以下の図\ref{fig:COM_candidate}に示すのが,上記の流れを経て残った
故障候補を確認できるコマンドを探索し,指標と共に提示した結果である.

\begin{figure}[H]
   \centering
      \includegraphics[width=12.0cm]{figure/COM_candidate.png}
      \caption{コマンドの選択肢表示}
      \label{fig:COM_candidate}
\end{figure}
\begin{figure}[H]
   \centering
      \includegraphics[height=8.0cm]{figure/COM_phase.png}
      \caption{コマンドによる確認}
      \label{fig:COM_phase}
\end{figure}
\end{comment}

%別で特定できなかった事例を取り上げて設計へのフィードバック.
%評価指標の優先順位を考えることでどのように検証結果が変化するのかを考察する.
\subsection{温度計故障に関する検証}%???
次に,以下の図\ref{fig:fault_mode2}に示すような
温度計故障(断線)を考え検証を行った例に関して述べる.
この時,異常検知の際の不具合事象としては,上の事例と同じく
\begin{itemize}
   \item 推進系ヒータONコマンド(ID:14)を送信したのに,推進系温度(ID:17)が上昇しない
\end{itemize}
という事象である.
テレメトリの確認や,初期コマンド状態からの確認情報の提示の流れは先ほどの例と
同様となる.
\begin{figure}[H]
   \centering
      \includegraphics[height=13.0cm]{figure/fault_mode2.png}
      \caption{故障箇所:リンク24(推進系温度計-TOBC間)の時の故障候補}
      \label{fig:fault_mode2}
\end{figure}
この時,システムによって洗い出された検証用のコマンドは上の例(図\ref{fig:COM_candidate})
のものと同じであり,指標から判断するとコマンド13の方が切り分けられるリンクの数は大きい一方で
全体のコマンド数の見積もりでは,コマンド18の方が少ないコマンド数であるという結果になっている.
コマンド13及び18を初めに選択した結果はそれぞれ以下の図\ref{fig:COM13_start},\ref{fig:COM18_start}
のようになった.
今回の不具合は温度計の断線であるため,パネルヒータによる推進系温度の変化も
推進系ヒータによる推進系温度変化も見ることはできないため,どちらも入力は異常(NG)を与えている.
最終的な結果は,どちらの過程を経ても同じになっており,
一つのリンクにまで故障箇所の特定を行うことができなかった.
このことから,この衛星モデルが今回扱った故障「(推進系)温度計故障」が発生した際に故障特定を行える設計
になっていないことが分かる.
このように,本手法は単に故障箇所の特定を支援するだけでなく,設計の不備を洗い出すことにも
利用できる.
\begin{figure}[H]
   \centering
      \includegraphics[height=8.0cm]{figure/COM14_TEL17_COM_COM_13start_mode2.png}
      \caption{コマンド13から選択した時の検証結果}
      \label{fig:COM13_start}
\end{figure}

\begin{figure}[H]
   \centering
      \includegraphics[height=8.0cm]{figure/COM14_TEL17_COM_COM_18start_mode2.png}
      \caption{コマンド18から選択した時の検証結果}
      \label{fig:COM18_start}
\end{figure}

本研究で構築したシステム上では故障箇所の特定までを行うことができなかったが,不具合分析の過程で得た情報
を用いて人間側で故障箇所の類推をすることは可能である.
今回の事例では,コマンドリンク20が正常であることは「パネル温度」によって確認できており,
パネル温度の上昇を確認することでヒータの正常が動作していることも確認できるため,推進系ヒータの故障ではない
ことが分かる.また,「パネルヒータON」によって推進系温度の変化を見ることができなかったことから,
テレメトリリンク:40,24が異常であることも類推できる.このように,本システムに従って検証を行うことによって,
故障箇所の推論するために必要な情報が取得可能であることがわかる.

実ミッションでは,設計段階においてFMEA(Failure Mode and Effect Analysis)などを用いて,
衛星システムに起こりうる故障モードを列挙し,それらの故障モードによる影響や,
発見のしやすさなどをもとに設計へのフィードバックを行う.よって,FMEA上で洗い出された
故障モードに対して本手法を適用することによって,それぞれの故障モードが発見可能な設計になっているか
を確認することができる.


\begin{comment}
まず,MOBCとTOBCからテレメトリは全て降ろされているものと仮定する.
また,故障は「推進系ヒータの接着不良」であるとする.
その時,テレメトリを通して確認できるのは
\begin{itemize}
   \item 「推進系ヒータON」コマンドを送ったのに,「推進系温度」テレメトリが変化しない.
\end{itemize}
という事象である.
よって不具合検知は,この事象によって行われる.

\subsection{不具合分析の例}
この不具合において,「推進系ヒータON」コマンドを送信して推進系ヒータに熱が伝わるまでの経路
と,推進系温度計が温度を読み取り「推進系温度」テレメトリとして地上局に伝わるまでの
経路の中に故障箇所があると考えることができる.
この経路を以下の図\ref{fig:simple_sat_fault}に太矢印で示している.
%これにリンクID入れたら見やすいかも
\begin{figure}[H]
   \centering
      \includegraphics[height=9.0cm]{figure/simple_sat_fault.png}
      \caption{故障箇所と不具合検知に関連するコマンドとテレメトリの経路}
      \label{fig:simple_sat_fault}
\end{figure}
また,この経路内にあるコンポーネントに電源が入っているかどうかを確認するためには,
そのコンポーネントに電源を供給するための経路が正常に作動しているかどうかを確認する
必要がある.そこで
\begin{itemize}
   \item PCU$\rightarrow$MOBC(リンクID:4.2),PCU$\rightarrow$TOBC(リンクID:5)
\end{itemize}
も検証を行う対象として考える.
以上より,検証すべき経路は下記のようになる.
\begin{itemize}
   \item コンポーネント:GS,MOBC,TOBC,Heater\_p,Prop,TempSensor\_p
   \item コマンドリンク:GS-MOBC(1),MOBC-TOBC(3),MOBC-PCU(4.2),
   PCU-TOBC(5),\\TOBC-Heater\_p(6),Heater\_p-Prop(10)
   \item テレメトリリンク:GS-MOBC(1),MOBC-TOBC(3),TOBC-TempSensor\_p(8),
   TempSensor\_p-Prop(14)
\end{itemize}
以下,提案手法によるアルゴリズムによって不具合分析を行っていく.\\
%この章長くて読みにくい
まず,問題設定よりMOBC及びTOBCのテレメトリは全てダウンリンクされている状態に
あるので,
\begin{itemize}
   \item テレメトリリンク:TOBC$\rightarrow$MOBC(3),MOBC$\rightarrow$GS(1)
   \item コマンドリンク:PCU$\rightarrow$MOBC(4.2),PCU$\rightarrow$TOBC(5)
\end{itemize}
は問題ないことが確認できるため,故障可能性はなくなる.
問題設定ではあるが,実際に「MOBC及びTOBCのテレメトリが全てダウンリンクされている」
ことを確認するためには,「MOBCカウンター」
「TOBCカウンター」を確認することが必要である.\\
また不具合発生時,推進系ヒータが正常に作動しており,
システム温度計からテレメトリを下ろす経路に問題がなければ,
「システム温度」が上昇しているはずである.
よって,テレメトリの確認によって検証できる経路として,
コマンドリンク「TOBC-Heater\_p(6)」がある.
以上より,不具合発生時から状態変化させずに確認すべき事項として以下
が挙げられる.

\begin{table}[H]
   \centering
   \caption{コマンドなしでの確認事項} 
   \label{tab:check_list1}
\end{table}
\vspace{-2zh}
\begin{figure}[H]
   \centering
      \includegraphics[height=2.5cm]{figure/check_list_tel.png}
\end{figure}
衛星の状態を変化させること無く,テレメトリを確認するだけで
検証できる箇所はこれ以上存在しないので,次にコマンドによる検証を行う.

まず,コマンドパスとしてGSに近い箇所から順に確認を行う必要がある.
MOBCへのコマンドが通っているかを確認するためには,「MOBCコマンドカウンターアップ」
コマンドを送って,「MOBCコマンドカウンター」が変化していることを確認できれば良い.
同様に考えると,以下の様に確認事項を洗い出すことができる.
\begin{table}[H]
   \centering
   \caption{コマンド送信による確認事項} 
   \label{tab:check_list2}
\end{table}
\vspace{-2zh}
\begin{figure}[H]
   \centering
      \includegraphics[height=2.6cm]{figure/check_list.png}
\end{figure}
以上の項目を確認した際,今回想定した故障モード(推進系ヒータの接着不良)
では期待されるテレメトリデータの変化が起こるので,
Table \ref{tab:check_list2}の確認対象パスの中にある故障可能性箇所は
棄却され,
故障可能性リンクとして残るのは
\begin{itemize}
   \item コマンドリンク:Heater\_p-Prop(10)
\end{itemize}
となる.

%上のコマンドを送っても大丈夫なのかという議論はどうするのか?もしシステム温度が高いのに,その方法でたしかめてもいいものなのか?

この経路上で考えうる故障モードと照らし合わせると,
この切り分けによって残る故障モードは
\begin{itemize}
   \item 推進系ヒータの故障
   \item 推進系ヒータの接着不良
\end{itemize}
となる.
この時,
「システム温度」の上昇によって「推進系ヒータの故障」の可能性は
棄却できるため,最終的に「推進系ヒータの接着不良」
が残り,
実際の故障を棄却すること無く,絞り込みができていると言える.\\
\end{comment}




\expandafter\ifx\csname ifdraft\endcsname\relax
  \end{document}
\fi