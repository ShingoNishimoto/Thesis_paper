\expandafter\ifx\csname ifdraft\endcsname\relax
\documentclass[11pt]{jsreport}
\usepackage{mypackage}
\begin{document}
\fi

\chapter{結論}

%今後の方針とかいるんかよ.どのように卒論にまとめていきたいか
%以上の不具合分析の流れを参考にし,探索のアルゴリズムを考えたい.
\section{今後の方針}
%専門家ばかりが関わっているわけではない,学生主導であるからという点も加える?
以上では,超小型衛星の信頼性向上の為の
不具合分析支援の手法に関して示し,テストケースに対する実践例を示した.
今後,いくつかの故障例を考えて実践し,
本手法を用いて不具合分析を行った結果と,指標を提示せず任意で
コマンドを選択した結果を比較し,
本手法の有効性を検証したいと考えている.
比較する際の評価軸としては
\begin{itemize}
   \item 効率的に不具合の切り分けが行えたかどうか(打ったコマンドの数で評価)
   \item 安全に切り分けを行うことができたかどうか(電力,姿勢の変化によって評価)
\end{itemize}
を考えている.
%評価軸を示して,この手法の有効性をどのように示す予定かをのべる.

\begin{comment}
\section{まとめ}

%これも今後の展望かな??
モデル化に関して
実際の衛星ではコンポーネント数やコマンド・テレメトリの数
が膨大であるためモデルが複雑化し,人によるモデル生成では
ヒューマンエラーや,作業量を考えると非現実的である.
そのため,将来的にはこれらを必要最低限の情報から生成する手法に関しても
検討していく.

%これも今後の展望かな
また,故障診断のコンテキストによってどこまで掘り下げるべきか使い分けるべきである
\cite{Ontology1998}ことを書く.

%モデルの波及効果に関してこれは今後の展望で述べるとこかな
一方で,人工衛星は内部のコンポーネントが非常に密集しているおり,人間が設計時に考慮した
意図したつながりだけでなく,意図しないつながりも多く存在する.
このような意図しないつながりによって,波及効果が発生することが衛星内部の理解が
困難になり
\end{comment}

\expandafter\ifx\csname ifdraft\endcsname\relax
  \end{document}
\fi