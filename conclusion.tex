\expandafter\ifx\csname ifdraft\endcsname\relax
\documentclass[11pt]{jsreport}
\usepackage{mypackage}
\begin{document}
\fi

\chapter{結論}

\section{まとめ}
本研究では,衛星内の情報伝達経路モデルを用いてコマンドによる故障箇所特定の過程を
体系化する手法,及びコマンドの安全性と故障候補切り分け能力の大きさを示す指標
を提案した.
%手法に関するまとめ的なものを入れられたらいいのかもしれないなあ
また,本手法を簡易的な衛星モデルで仮想的に与えた故障状態に対して適用し,
本手法の評価を行った.\\
実践例では,複数のコマンドの中から故障箇所特定のために適切なコマンドを探索し,
そのコマンドに対して評価指標の計算を行ったものと共に提示した.
提示されたコマンドを用いて検証を行うことにより,想定した故障箇所を
特定できる能力があることを示した.\\
次に,コマンドの故障候補切り分け能力を示す指標に関して考察を行った.
そこでは,時間制約がより厳しい状況での不具合分析においては「平均確認可能性」の大きなコマンドを用いる
ことで数少ない通信の機会を利用できる可能性が高まること,
不具合分析に使用できる時間が明確に分かっており,ある程度の余裕がある場合には
「検証コマンド総数」が小さなコマンドを選択することで,最終的に少ない作業工程で
故障箇所の特定が行える可能性があることを示した.\\
%もう少し人との推論で特定できる,そのための情報を効率的に集めることができることを伝える.
また,故障状態によってはシステムのみでは特定を行えなかった場合があること
もわかった.そのような場合には人間の推論と組み合わせることで故障候補の絞り込みができ,
提示された選択肢に従うことで推論に必要な情報を効率的に得ることができることを示した.
同時に,人との推論を組み合わせても特定できない場合には設計の不備を考えられるため,
設計の不備を発見することにつながることを示した.

一方で,本手法では
安全性を示す指標として状態変化の小さなものが好ましいと考えていたが,
人による不具合分析結果との比較によって,実際は安全を確保するために状態変化をする場合がある
こと,本手法で見ることができる故障は主に接続関係に関するものに限定されており,
コンポーネントが持つ機能の故障まで特定することはできないことなどが知見として得られた.\\
これらを踏まえた今後の展望を次に示す.

%テレメトリを発行している機器の故障の場合は,そのコンポーネントから
%の情報ラインに冗長系がなければかなり多くの故障候補が残ってしまう.
%多分複数の検証結果を総合して判断するような処理が必要になるんやろな.これをシステム上で判断させること
%ができれば実現できるのかもしれないが,ムズいぜ
\newpage
  \section{今後の展望}
  今後取り組むべき課題として以下にまとめた.
\begin{itemize}
  \item コンポーネントの機能の接続関係を組み込んだ,より粒度の細かい故障箇所特定
  \item リンクの正常確率に実機の情報を組み込むことによる,検証の効率化
  \item 設計情報からのモデル自動生成
  %\item コマンドの安全性の評価・・・
  %\item 推論も
\end{itemize}
上記の課題に関して,今後進めていくべき具体的な取り組みを述べる.

まず,コンポーネントの接続関係の故障だけでなく,各コンポーネントの機能の故障
を扱うために,各コンポーネントが持つ機能の詳細なモデル化が必要になる.
コンポーネントの機能は,階層的になっておりある機能を満たすためのサブ機能が
存在するというような関係になっている.検証を行う過程に関しても
まずは粗い粒度で故障箇所の特定を行い,その後故障箇所コンポーネントの
持つ機能単位での故障の特定,サブ機能単位での故障の特定という風に段階的に
切り分けを行っていく必要があると考えている.\\
また,上述した機能に対する故障を見るためにはテレメトリが持つ情報の意味に関しても
システムが扱えるようなオントロジーを定義する必要がある.
現在,人間からの入力情報として,テレメトリが正常か異常かの2値しか与えることができていない.
実際には,テレメトリの種類によって正常か異常かの基準はいくつかあり,
 %このまとめ方も微妙な気がする.
\begin{itemize}
  \item[-] テレメトリの取得可否
  \item[-] テレメトリに含まれるパラメータの大小
  \item[-] テレメトリの変化の有無
\end{itemize}
などが考えられる.
これらの情報の違いを扱うために
「テレメトリの種別」という概念を導入し,人間による入力のパターンを増やすことで,
故障の種類を見分けることができるようにする必要がある.
また同時に,機能に対応した故障を特定するためには,テレメトリと状態量の対応付けを考える必要があり
このモデルをどのように構築するかに関しては今後検討を進めていく必要がある.\\

 また,本研究では各リンクの正常確率を簡単のため0.5と固定して与えた.
 実際の衛星開発では,その機関で長年培われた技術や,実績のある機器など信頼性の高い
 設計項目が存在し,同時に新規実装項目や開発途中のソフトウェアなど信頼性の低い設計箇所
 も存在するなど,ばらつきがある.そこで,
 リンクの正常確率を,実際に設計・製造している衛星の各設計項目に対する
 信頼度に基づいて考えることによって,より効率的な検証作業につながることが想像できる.
これらの信頼度は事前知識的に組み込むことは可能であるが,今後の課題としては
試験結果を元に不具合発生するコンポーネントの信頼性を下げていくなどし,学習させる
システムを構築することによって,対象とする衛星のモデルの再現度を高めることを検討したい.\\


また,本研究で用いた簡易的な衛星モデルは全て手作業によって記述した.
実際に開発される衛星では,超小型衛星であっても膨大な数のコマンド及びテレメトリ,%具体的な数ほしい
多くのコンポーネント間の回路などが存在し,手作業によるモデルの記述は作業コストや
ヒューマンエラーのリスクを考えると現実的ではない.また,実際の
開発過程においては設計変更が幾度に渡って繰り返されることがほとんどである.
そのため,その設計変更との整合性が取れなくなると本手法を用いた不具合分析を行うことは
不可能である.これらを踏まえると,設計情報からモデルを自動生成し,
設計情報の更新に応じて本手法で用いるモデルにも反映されるシステムが求められる.
設計情報の中には,各コンポーネントの電気回路の接続情報や,物理的位置関係
など,本手法で用いたモデルを構築するために必要な情報が含まれている.
それらの文書の中からモデル生成を行う手法に関して,検討していこうと考えている.



\begin{comment}
\section{まとめ}

%これも今後の展望かな??


%これも今後の展望かな
また,故障診断のコンテキストによってどこまで掘り下げるべきか使い分けるべきである
\cite{Ontology1998}ことを書く.

%モデルの波及効果に関してこれは今後の展望で述べるとこかな
一方で,人工衛星は内部のコンポーネントが非常に密集しているおり,人間が設計時に考慮した
意図したつながりだけでなく,意図しないつながりも多く存在する.
このような意図しないつながりによって,波及効果が発生することが衛星内部の理解が
困難になり
\end{comment}

\expandafter\ifx\csname ifdraft\endcsname\relax
  \end{document}
\fi