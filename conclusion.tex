\expandafter\ifx\csname ifdraft\endcsname\relax
\documentclass[11pt]{jsreport}
\usepackage{mypackage}
\begin{document}
\fi

\chapter{結論}

\section{本研究で得られた知見}%知見なのか?まとめ的な
本研究ではモデルを用いてコマンドによる故障箇所特定のプロセスを体系化する手法を提案し,
テストケースを用いてその有効性を検証した.

  本手法を用いて最終的な故障箇所の特定を行うのは難しい.どちらかというと
  不具合分析過程を体系化して,それを用いたコマンドの選択をすることによって
  故障箇所の推論に必要な情報を集めるような働きをしていると言える.

  テレメトリを発行している機器の故障の場合は,そのコンポーネントから
  の情報ラインに冗長系がなければかなり多くの故障候補が残ってしまう.
%多分複数の検証結果を総合して判断するような処理が必要になるんやろな

  \section{今後の展望}
 今後,接続関係の異常だけでなく,実問題に近い故障状態も扱えるようにするために,
 扱う状態量をより詳細にモデル化していく必要がある.また,テレメトリと状態量の対応付けを考えることによって
 異常状態をリンクとして表現するのではなく,各コンポーネントの機能の異常を特定できると考えている.

 また本手法では,簡易的に故障候補の洗い出しを情報伝達の経路のみに絞っていたが,
 \cite{}で提案されているオントロジーを用いることで,細かい粒度の特定を目標に行っていく必要がある.

 今回の例では,モデルの構築を手作業により行ったが,実ミッションでの適用を考慮すると手作業によるモデル構築は
 現実的ではない.今後ある程度の事前定義情報からモデル化を自動化することを目標にする.
  
 リンクの正常確率として,より実機で用いているコンポーネントの信頼度に近い値を考えることで,
 モデルが複雑になった際により効率的な故障箇所特定を行えると考えている.


%今後の方針とかいるんかよ.どのように卒論にまとめていきたいか
%以上の不具合分析の流れを参考にし,探索のアルゴリズムを考えたい.
\section{今後の方針}
%専門家ばかりが関わっているわけではない,学生主導であるからという点も加える?
以上では,超小型衛星の信頼性向上の為の
不具合分析支援の手法に関して示し,テストケースに対する実践例を示した.
今後,いくつかの故障例を考えて実践し,
本手法を用いて不具合分析を行った結果と,指標を提示せず任意で
コマンドを選択した結果を比較し,
本手法の有効性を検証したいと考えている.
比較する際の評価軸としては
\begin{itemize}
   \item 効率的に不具合の切り分けが行えたかどうか(打ったコマンドの数で評価)
   \item 安全に切り分けを行うことができたかどうか(電力,姿勢の変化によって評価)
\end{itemize}
を考えている.
%評価軸を示して,この手法の有効性をどのように示す予定かをのべる.

%こんなんできていないので今後の課題として提示するにとどめる.
また,これらの信頼性を試験結果から学習させることによって,対象とする衛星に対する
モデルの再限度を高め,効率的な不具合分析を行うことが可能になる.

\begin{comment}
\section{まとめ}

%これも今後の展望かな??
モデル化に関して
実際の衛星ではコンポーネント数やコマンド・テレメトリの数
が膨大であるためモデルが複雑化し,人によるモデル生成では
ヒューマンエラーや,作業量を考えると非現実的である.
そのため,将来的にはこれらを必要最低限の情報から生成する手法に関しても
検討していく.

%これも今後の展望かな
また,故障診断のコンテキストによってどこまで掘り下げるべきか使い分けるべきである
\cite{Ontology1998}ことを書く.

%モデルの波及効果に関してこれは今後の展望で述べるとこかな
一方で,人工衛星は内部のコンポーネントが非常に密集しているおり,人間が設計時に考慮した
意図したつながりだけでなく,意図しないつながりも多く存在する.
このような意図しないつながりによって,波及効果が発生することが衛星内部の理解が
困難になり
\end{comment}

\expandafter\ifx\csname ifdraft\endcsname\relax
  \end{document}
\fi