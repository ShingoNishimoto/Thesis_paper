\expandafter\ifx\csname ifdraft\endcsname\relax
\documentclass[11pt]{jsreport}
\usepackage{mypackage}
\begin{document}
\fi


\chapter*{謝辞}
\addcontentsline{toc}{chapter}{謝辞}
%研究としたほうがいいのか?
本論文は筆者が東京大学工学部航空宇宙工学科での卒業研究の成果をまとめたものである.\\
指導教員の船瀬龍准教授には,ご多忙の中研究相談にのって頂き,私の研究方針が定まらない段階からも
取り留めのない内容の私の話に親身に耳を傾けて下さり,終始ご指導を頂いた.
ここに深謝の意を表する.

中須賀真一教授には,私とのミーティングを何度も快く受けて下さり,その度に
私の研究に対する漠然としたイメージを適切な言葉で表現し,新たな気付きと
多くのアイデアを頂いた.%ちょっと違う
ここに深謝の意を表する.

五十里哲助教には,輪講の場や資料の添削を通じて自身の豊富な衛星開発経験の知見を元にした
数々の鋭いご指摘を頂くと共に,資料作成の細部にわたりご指導を頂いた.
ここに感謝の意を表する.

小畑俊裕共同研究員はご多忙の中,私とのミーティングを快く受けて下さり,私の浅薄な研究アイデア
に対してご自身の知見を元に多くのご指摘とご提案をして頂いた.また,私の選択した研究テーマ
に興味を持って下さり,有難い応援の言葉を頂いた.
深く御礼申し上げる.

横堀慎一研究員には,民間企業での衛星開発の経験を元にシステムズエンジニアリング
的な取り組みや,大規模な衛星開発における課題に関する話をして下さり,研究の種となるような
気付きを与えて頂いた.心より感謝申し上げる.

松本健研究員にはミーティングの場で,自身の衛星開発及び運用経験に基づき,様々な過去の不具合分析の事例を取り上げ,
不具合分析における課題をご説明頂いた.また,全てのお話の中で衛星開発に関して
無知な私が理解するまで丁寧にご説明頂いた.心より感謝申し上げる.

中島晋太郎共同研究員には,研究グループ発表時のコメントや
個別のミーティング,高橋亮平さんとのミーティングなど数多くの場面を通して
私の研究への深いご理解と丁寧なご指導を頂いた.
厚く御礼を申し上げる.

高橋亮平さんには,自身の研究やプロジェクトでの立場上ご多忙にも関わらず,
私の研究相談を毎週開いて下さり,進捗が生めていない中でも現状を聞き,
終始ご指導頂いた.また,研究者として未熟な私に対して,
文書作成や輪講での発表に対する助言を頂き,研究の方向性が定まらなかった
私を正しい方向に導いて下さった.
ここに深謝の意を表する.

また,3章で述べた不具合分析に関する調査にご協力下さった中須賀・船瀬研究室の皆様,
新型コロナウイルス感染症の影響により研究室での研究活動を行うことが困難な中,私に
素晴らしい環境を与えて下さった研究室関係者各位に改めて感謝する.

最後に,この3年間の東京での不自由ない生活を支援し続け,
このような学習の機会を与えて下さった,家族に深謝する.


\expandafter\ifx\csname ifdraft\endcsname\relax
  \end{document}
\fi