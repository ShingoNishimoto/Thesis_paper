\expandafter\ifx\csname ifdraft\endcsname\relax
\documentclass[11pt]{jsreport}
\usepackage{mypackage}
\begin{document}
\fi

\chapter{序論}
%ここたぶん要らん
\section{はじめに}
現在,コマンドとテレメトリを用いた衛星の故障箇所の特定を
支援する手法を検討している.\\
以下の章では,まず研究背景として地上試験におけるリスク分析の不十分さ及び,
不具合原因仮説の検証を支援する研究が十分に行われていないことを述べ,次にそれを踏まえた
研究目的に関して述べる.
また,提案手法の章では,不具合分析のアルゴリズム,
使用するモデルに関して述べ,そのモデルを用いた仮説検証のための
コマンド及び確認事項の探索方法,人がコマンド選択をする際に必要な
評価指標に関して説明する.
最後に,簡易衛星モデルを用いた実践例を示し,今後の方針について述べる.

\section{研究背景} 
\subsection{超小型衛星の信頼性の低さ}
近年,
超小型衛星の開発が大学や小企業の中で盛んになってきている.
これまでは教育目的が主であったが,商用利用や革新的なミッションへの応用も
増えてきている\cite{Langer2016}.
一方で現状の超小型衛星は中・大型衛星と比較して軌道上での不具合の確率は高く,
2002から2016の間に打ち上がった
270のCubesatのうち,139のミッションが失敗している\cite{Langer2016}.\\
大学衛星は宇宙環境での使用を保証されていない
民生部品を使用することも多いため,このような超小型衛星で頻発している
不具合は,軌道上での部品の故障によって発生すると考えられてきた.しかし,
実際には多くが設計や製造過程に起因する
%故障分析の意味
不具合であることが故障分析を通じて知られている\cite{Venturini2017}.
軌道上での不具合の根本原因に対する調査(図\ref{fig:cause of failure})では,
民生部品の品質の不確定性が原因であったものはわずか17%であり,
それ以外の多くが設計や,地上試験の不足に起因するものである\cite{Venturini2017}.

%ここにできれば具体的な衛星の故障の例を持ってくれると良い
%論文で
\begin{figure}[H]
   \centering
      \includegraphics[height=4.5cm]{figure/cause_of_failure.png}
      \caption{故障原因に関するインタビュー結果\cite{Venturini2017}}
      \label{fig:cause of failure}
\end{figure}

%ここへのつながり.別subsectionのほうがいいかも
%ほどよし信頼性工学のところをしっかりとまとめたほうがいい
また,大学衛星が商用利用や革新的なミッションに
挑戦するためには,超小型衛星のメリットである
コストの低さを十分に確保しながら,ほどほどの信頼性
を実現する「ほどよし」の考え方が
重要であると考えられている\cite{SHIRASAKA2011}.\\
故障に設計や製造の不良が含まれていることを考えると,
超小型衛星のほどほどの信頼性の評価を行うためには,
従来用いられてきた
各コンポーネントごとの信頼度の組み合わせでは不十分である.
そこで,設計・製造・運用における
信頼度を加味した評価手法が提案されている\cite{SHIRASAKA2011}.
式(\ref{eq:Reliability})が示すように,この評価手法では
設計や製造時の信頼性も重要な要素であると捉えられている.\\
設計や製造における信頼性を高めるためには,地上試験を通じて
それらの過程で発生した不備・不具合を改修することが必要である.

コストを考えると,信頼性の高いコンポーネントを使用することで
$R_{comp}$を高めるより,設計や製造過程における信頼性を高めることが超小型衛星
の信頼性の向上につながる.

\begin{equation}
   R_{sat} = R_{des} \times R_{fab} \times R_{comp} \times R_{op} \label{eq:Reliability}
\end{equation}
\begin{table}[H]
   \centering
      \begin{tabular}{cl} 
        $R_{sat}$ & 衛星の真の信頼度\\
        $R_{des}$ & 設計における信頼度\\
        $R_{fab}$ & 製造における信頼度\\
        $R_{comp}$ & 衛星の信頼度(従来の信頼度)\\%コンポーネントの信頼度
        $R_{op}$ & 運用における信頼度
      \end{tabular}
\end{table}
%これを言うことで,この研究ではこれらのどこを高めているのかを考える必要がある.

%もうちょいちゃんと考える
\subsection{地上試験における問題}
以上で示したように,不具合の多くが設計,製造などに起因している
という問題がある.
一方で,これは超小型衛星開発のみに限られたことではなく,
中・大型衛星においても大きな問題となっている.
軌道上故障データを分析した結果\cite{SAITO2011}(図\ref{fig:error type})
によると,軌道上で
偶発的に発生した故障はわずか11%であり,それ以外は設計,製造などの開発
活動に起因するものであることが分かっている.\\
また,軌道上で発生した不具合が「地上試験で
発現しなかった,または発見できかった原因」が以下の
図\ref{fig:error cause}のように知られている.
試験設備の不足によるものや,故障発見までの
時間が長く試験で発見することが現実的で無いものに関しては,
コストとリソースの面から試験による対策では限界がある.
一方で,試験モードの不備や,発現していたのにもかかわらず
発見できなかった不具合に関しては試験に対する習熟度が不足していること,
不具合・リスクの分析が不十分であることが推測される\cite{SAITO2011}.

\begin{figure}[H]
   \centering
      \begin{tabular}{c}
         \begin{minipage}{0.50\hsize}
         \centering
         \includegraphics[width=5.5cm]{figure/on_orbit_error_tyoe.png}
            \caption{軌道上故障の原因類型の分布\cite{SAITO2011}}
            \label{fig:error type}
         \end{minipage}
         \begin{minipage}{0.50\hsize}
         \centering
         \includegraphics[height=5.5cm]{figure/not_found_error_seeds.png}
            \caption{軌道上故障の要因を地上で発見できなかった原因類型の分布\cite{SAITO2011}}
            \label{fig:error cause}
         \end{minipage}
      \end{tabular}  
\end{figure}

%ここの流れを見直す.根拠資料
\subsection{不具合原因特定の難しさ}
以上のように,衛星の不具合及びリスク分析を,地上試験で十分に
行うことができていないことが,超小型衛星の信頼性の低さの原因の一つであった.\\%言い回し
そこで,地上試験で衛星の不具合及びリスク分析が十分に行えていない原因
を具体的に示すため,以下に人間による不具合原因分析の大まかな流れを示す.
\begin{enumerate}[1)]
   \item 不具合が起きた際の衛星の状態を保存し記録に残す. 
   \item テレメトリから考えられる故障原因の候補を洗い出す.
   \item それらの故障の中でテレメトリから分かる情報を元に候補を棄却していく.
   \item 更に切り分けが必要な場合はコマンドを送り,
   それに対するテレメトリの挙動によって判断するという作業を繰り返す.
   \item 判断できない場合は,コンポーネントを取り出し直接確認を行う.
\end{enumerate}
まず,2)の故障原因の候補
の洗い出しを網羅的に行うことの難しさがある.\\
組み上げ状態の衛星から得られる情報はテレメトリのみである.
この際,衛星の内部状態を理解し,テレメトリから現在の衛星の状態
を想像することができなければ,十分に不具合原因の候補を洗い出すことはできない.
本研究室の過去プロジェクト(PRISM)を対象にした研究では,
事前に想定していた故障モードの粒度は,%故障モードの意味
%ここの記述も微妙なので変更する
山口ら\cite{Yamaguchi2014}が構築したシステム
を用いて洗い出したものと比べて,不足しているという結果も出ている.
このように,人間による故障モードの洗い出しは思いつきによるものなので,
考えが及んでいないことが多い.

また,分析が不十分になっているもう一つの原因として,
3),4)の故障原因の切り分け作業の難しさもある.\\
超小型衛星は内部状態が複雑に絡み合っており,一つの不具合に対して
非常に多くの故障候補が洗い出されることが想像できる.
%切り分けを行うためのコマンドを探すのが難しい
そのため,多くの故障候補の中から切り分けを行い,最終的な故障を
特定するという作業は多くの知識と労力を必要とする作業である.
また,実ミッションで使用するコマンドとテレメトリは膨大な数であるため,
その中から切り分けを行うための情報を選択し,仮説の検証を行う作業は
無駄やヒューマンエラーを生むきっかけとなる.
故障仮説を検証する際,未熟な運用者が不具合原因特定のために誤った%ここの表現改める
コマンドを送信してしまうと,衛星の生存を脅かす可能性がある.
このため,不具合原因特定を行う際には
そのコマンドが”安全”なのかという,衛星の状態に依存した点も非常に重要となる.

%切り分け作業の難しさもだが,先行研究の事例を示し,故障候補の洗い出しに関しては
%広く検討されていることを述べて,自分はこちら側をやるという方向で示せばいいのでは?

%それはそうだが,下の流れを作るためには用途に応じてコマンド選択の指標が変わることを説明したい

\subsection{不具合分析に関する先行研究}
上述のように,不具合原因の洗い出し
が網羅的にできていないこと,コマンドとテレメトリを用いて
原因特定を行う過程が知識依存になっていること
が,不具合分析が不十分になっている原因の一つであった.%一つでないが
これらの課題に対して,古くから%古くから???
不具合分析システムの研究が盛んに行われている.
以下のTable \ref{tab:previous_research}に,モデルベースで
機械などを対象にした不具合分析,故障診断を行う手法に関してまとめた.
%比較軸が微妙過ぎる
\begin{table}[H]
   \centering
   \caption{不具合分析手法の比較}
   \label{tab:previous_research}
      \begin{tabular}{cccccc} \hline%もう少し示し方を考える.
         手法&故障網羅性&手法の目的%&モデル複雑度%専門家の知識が必要という点で?
         \\ \hline
         GDE&低&故障仮説生成%&低
         \\ %見てないし無くてもいいかも
         GDE+\cite{Struss1989}&中&故障仮説生成%&中
         \\
         網状故障解析\cite{Yamaguchi2014}&中&異常モード洗い出し%&高
         \\
         故障オントロジー\cite{Kitamura1999}&高&故障仮説生成%&高
         \\
         本手法&中%低かもしれない.接続関係しか見れていない
         &故障箇所特定%&中
         \\ \hline
      \end{tabular}
\end{table}
%ここで本手法を出すのは適切なのか?何か分からんくない?
%モデル複雑度が比較の指標になるのか?
%モデルベースで行う不具合分析手法として一般的なものは
故障仮説生成の研究に関しては,
機器の正常時のモデルだけでなく,故障時モデルを組み込んだもの\cite{Struss1989}や,
オントロジーを用いてプロセスのつながりまでモデル化したもの\cite{Yamaguchi2014},
異常伝播事象までモデル化して階層的な推論を行うもの\cite{Kitamura1999}などが
%不具合原因である機器の故障の原因などを探索可能にしたものなどが
あり,網羅的に故障候補を洗い出すために広く取り組まれている.
一方で,來村ら\cite{Kitamura1999}が効率の良い検証方法に関しては
今後の課題として言及しているように,故障仮説の検証に取り組んだものは少ない.
%何か足りん気がする
故障候補の洗い出しを十分に行うことができたとしても,特定を行うことができなければ
地上試験によって設計や製造における不備を除くことができない.

\section{研究概要}
\subsection{本研究での目的}
以上を踏まえると,
不具合発生時に故障候補を洗い出し,その中から
原因を特定していく過程に,高い知識と経験が必要である
ことが,衛星の不具合やリスクの分析が不十分に
なっている原因の一つであると推察される.
また,故障候補の網羅的な洗い出し
に関しては広く取り組まれている一方で,仮説の検証作業の支援に関して
取り組んだ研究は行われていない.

%機能に関しても箇条書きで分かりやすくできんか?
そこで本研究では,経験が浅く,衛星に関する知識の乏しいエンジニアであっても,
不具合事象から故障箇所の特定を行えるような不具合分析支援手法の提案を目的とする.
また,以下では不具合発生から故障箇所の特定を行う過程を「不具合分析」と表現している.\\

\begin{itemize}
  \item 不具合発生時に故障箇所を特定するために,確認すべきテレメトリ,打つべきコマンド
  を選択肢として提示すること
  \item コマンドの選択肢を選ぶ際の判断の指標を定量的に示すこと
  \item システムに対して人間がテレメトリから解釈した衛星の状態をフィードバックすることで
  システムに求められる忠実度を下げること.%????
\end{itemize}



以上の機能を満たすために,本手法は下記の3つの要素で構成されている.
\begin{itemize}
   \item 衛星内部のコンポーネント間接続関係モデル及び,情報伝達経路モデル
   \item 故障箇所の特定を行うために必要なコマンド及びテレメトリの探索 %ここの言い回し
   \item 人間の判断を支援するコマンドの評価指標の提案
\end{itemize}

\subsection{本論文の構成}
本論文は次のような構成となっている.
第2章では,本研究で提案する不具合分析手法で使用する衛星のモデル
と不具合分析のアルゴリズム,結果を評価するための指標に関して述べている,
また,第3章では提案手法を実践した結果をに関してまとめている.
%章番号と一致しているのかを確認する.

\expandafter\ifx\csname ifdraft\endcsname\relax
\end{document}
\fi